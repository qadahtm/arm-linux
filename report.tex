\documentclass[10pt]{article}
\usepackage{amsmath}
\usepackage{amssymb}
\usepackage{amsthm}
\usepackage{amscd}
\usepackage{amsfonts}
\usepackage{graphicx}%
\usepackage{fancyhdr}

\usepackage{mathpazo}
\usepackage[left=1.2in, right=1.2in, top=1.2in, bottom=1in]{geometry}

\theoremstyle{plain} \numberwithin{equation}{section}
\newtheorem{theorem}{Theorem}[section]
\newtheorem{corollary}[theorem]{Corollary}
\newtheorem{conjecture}{Conjecture}
\newtheorem{lemma}[theorem]{Lemma}
\newtheorem{proposition}[theorem]{Proposition}
\theoremstyle{definition}
\newtheorem{definition}[theorem]{Definition}
\newtheorem{finalremark}[theorem]{Final Remark}
\newtheorem{remark}[theorem]{Remark}
\newtheorem{example}[theorem]{Example}
\newtheorem{question}{Question} \topmargin-2cm

%\textwidth6in

\setlength{\topmargin}{0in} \addtolength{\topmargin}{-\headheight}
\addtolength{\topmargin}{-\headsep}

%\setlength{\oddsidemargin}{0in}

%\oddsidemargin  0.0in 
%\evensidemargin 0.0in 
\parindent 0em
\parskip 1em

\pagestyle{fancy}\lhead{Visualizing Memory Pages} \rhead{Yiyang Chang and Thamir Qadah}
\chead{{\large{\bf }}} \lfoot{} \rfoot{ } \cfoot{\thepage}

\newcounter{list}


\newcommand{\HRule}{\rule{\linewidth}{0.5mm}}
\newcommand{\Hrule}{\rule{\linewidth}{0.3mm}}

\makeatletter% since there's an at-sign (@) in the command name
\renewcommand{\@maketitle}{%
  \parindent=0pt% don't indent paragraphs in the title block
  
  \begin{flushright}
  \@author \\ \@date
  \end{flushright}
  
  \centering
  {\Large {\@title}}
  %\HRule\par%
  %\vspace{0.1in}
  \par
}
\makeatother% resets the meaning of the at-sign (@)






%%%%%%%%%%%%%%%%%%%%%%%%%%%%%%%%%%%%%%%%%%%%%%%%%%%%%%%%%%%%%

%%%%%%%%%%%%%%%%%%%%%%%%%%%%%%%%%%%%%%%%%%%%%%%%%%%%%%%%%%%%%
\title{ECE695 Programming Assignment 2: Visualizing Memory Pages}
\author{Yiyang Chang and Thamir Qadah}
\date{chang256@purdue.edu, tqadah@purdue.edu}

\begin{document}
%\raisebox{1cm}

\maketitle

\section{Overall Design}
\vspace{-0.2in}
We modify linux kernel for ARM to support visualizing memory page reference
counts. We augment linux /proc file system to display a number for each memory
page. The counting is enabled by calling /proc/PID/maps, and the reference
counts are maintained up to 9 and x for counts larger than 9 for the sake of
clean display. 

The counting is done inside linux exception handlers. We use do\_DataAbort for
data pages and do\_PrefetchAbort for code pages. Page monitoring is turned on with
the fist read of /proc/PID/maps. At this time, we invalidate all hptes
mapped from each virtual address and trigger exceptions. Once invalidated, we
trap the page fault based on the type of pages, and update a counter for
that page. 

For each process, a linked list containing the page access counts for all monitored page 
table entries (ptes). We also, keep other information such as the following:

\vspace{-0.2in}
\begin{itemize}
	\item Hardware ptes in order to restore them when unmasking.
	\vspace{-0.1in}
	\item Page type: this refers if the page is a code page, a data page or others. 
	\vspace{-0.1in}
	\item Saved PC and saved patched instruction for trapping user process after 
	\vspace{-0.1in}
unmasking. 
	\end{itemize}


\section{Features}
\vspace{-0.2in}
\begin{itemize}
	\item Turn on reference counting by calling /proc/PID/maps. 
	\vspace{-0.1in}
	\item Reference counting (0-9, x for above 9) for memory pages
	\vspace{-0.1in}
	\item Swap following instructions to generate undefined instruction
		exception for both ARM and thumb instructions 
	\vspace{-0.1in}
	\end{itemize}

\section{Interesting Implementation}
\vspace{-0.2in}
\begin{itemize}
	\item Maintain the reference counting in a linked list in task\_struct. This 
allows us to maintain large counters up to . However, this will impact the one digit 
visualization specs. 
	\vspace{-0.1in}
	\item Distinguish between ARM and thumb instructions when patching the next 
instruction to trap the user process execution. However, since Thumb instructions 
requires decoding, patching is not implemented and thus disable monitoring for that 
particular page. 
	\vspace{-0.1in}
	\item We also introduce additional legends for virtual page visualization with 
/prog/PID/maps. In particular, for pages that are not monitored, we display "n" as an 
indicator for no monitoring. 
	\vspace{-0.1in}
	\item Through debugging, we realize that one instruction could trigger
	both data abort and prefetch abort, which results in patching the
	following instruction multiple times. This needs to be carefully
	handled because it does not make sense to restore an undefined
	instruction.
	\vspace{-0.1in}
\end{itemize}

\section{Known Issues and Bugs}
\vspace{-0.2in}
\begin{itemize}
	\item We monitor only specific vmas such as code segment, data, segment, heap and 
stack. 	    
	\vspace{-0.1in}
	\item Sometimes kernel crashes when library pages are being monitored
	with (kernel panic - not syncing: Attempted to kill init) but counts
	correctly. By default, tracking library pages needs to be turned on
	explicitly in the code. (ref. L405 in ``fs/proc/task\_mmu.c'')
	\vspace{-0.1in}
	\item Some counts are not reflected immediately. 
	\vspace{-0.1in}
	\item Only do counting for a predefined process. 
	\vspace{-0.1in}
    
\end{itemize}

\end{document}
