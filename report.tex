\documentclass[11pt]{article}
\usepackage{amsmath}
\usepackage{amssymb}
\usepackage{amsthm}
\usepackage{amscd}
\usepackage{amsfonts}
\usepackage{graphicx}%
\usepackage{fancyhdr}

\usepackage{mathpazo}
\usepackage[left=1.4in, right=1.4in]{geometry}

\theoremstyle{plain} \numberwithin{equation}{section}
\newtheorem{theorem}{Theorem}[section]
\newtheorem{corollary}[theorem]{Corollary}
\newtheorem{conjecture}{Conjecture}
\newtheorem{lemma}[theorem]{Lemma}
\newtheorem{proposition}[theorem]{Proposition}
\theoremstyle{definition}
\newtheorem{definition}[theorem]{Definition}
\newtheorem{finalremark}[theorem]{Final Remark}
\newtheorem{remark}[theorem]{Remark}
\newtheorem{example}[theorem]{Example}
\newtheorem{question}{Question} \topmargin-2cm

%\textwidth6in

\setlength{\topmargin}{0in} \addtolength{\topmargin}{-\headheight}
\addtolength{\topmargin}{-\headsep}

%\setlength{\oddsidemargin}{0in}

%\oddsidemargin  0.0in 
%\evensidemargin 0.0in 
\parindent 0em
\parskip 1em

\pagestyle{fancy}\lhead{Visualizing Memory Pages} \rhead{Yiyang Chang and Thamir Qadah}
\chead{{\large{\bf }}} \lfoot{} \rfoot{ } \cfoot{\thepage}

\newcounter{list}


\newcommand{\HRule}{\rule{\linewidth}{0.5mm}}
\newcommand{\Hrule}{\rule{\linewidth}{0.3mm}}

\makeatletter% since there's an at-sign (@) in the command name
\renewcommand{\@maketitle}{%
  \parindent=0pt% don't indent paragraphs in the title block
  
  \begin{flushright}
  \@author \\ \@date
  \end{flushright}
  
  \centering
  {\Large {\@title}}
  %\HRule\par%
  %\vspace{0.1in}
  \par
}
\makeatother% resets the meaning of the at-sign (@)






%%%%%%%%%%%%%%%%%%%%%%%%%%%%%%%%%%%%%%%%%%%%%%%%%%%%%%%%%%%%%

%%%%%%%%%%%%%%%%%%%%%%%%%%%%%%%%%%%%%%%%%%%%%%%%%%%%%%%%%%%%%
\title{ECE695 Programming Assignment 2: Visualizing Memory Pages}
\author{Yiyang Chang and Thamir Qadah}
\date{chang256@purdue.edu, tqadah@purdue.edu}

\begin{document}
%\raisebox{1cm}

\maketitle

\section{Overall Design}
\vspace{-0.1in}
We modify linux kernel for ARM to support visualizing memory page reference
counts. We augment linux /proc file system to display a number for each memory
page. The counting is enabled by calling ``/proc/PID/maps'', and the reference
counts are maintained up to 9 and x for counts larger than 9 for the sake of
clean display. 

The counting is done in linux exception handlers. We invalidate all the hptes
mapped from each virtual address and trigger data abort exception, and capture
the exception to accomplish reference counting.

\section{Features}
\vspace{-0.1in}
\begin{itemize}
	\item Turn on reference counting by calling ``/proc/PID/maps''
	\item Reference counting (0-9, x for above 9) for memory pages
	\item Swap following instructions to generate undefined instruction
		exception for both ARM and thumb instructions 
	\end{itemize}

\section{Interesting Implementation}
\vspace{-0.1in}
\begin{itemize}
	\item Maintain the reference counting data structure in task\_struct
	\item Take care of both ARM and thumb instructions in instruction patching
\end{itemize}

\section{Known Bugs}
\vspace{-0.1in}
\begin{itemize}
	\item Does not work for code pages and library pages
	\item Only does counting for a given process
\end{itemize}

\end{document}
